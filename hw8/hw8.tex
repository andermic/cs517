\documentclass{article}
\usepackage[pdftex]{graphicx}
\usepackage{amsmath}
\usepackage{verbatim}
\usepackage{enumerate}
\usepackage{parskip}
\author{Michael Anderson}
\title{Homework 8}
\begin{document}
\maketitle
\center{CS517}
\center{Prof. Cull}\\
\flushleft
\newpage

\section{}
Does:

\[
\frac{a}{b} \le \frac{p}{q} \hspace{1em} \text{and} \hspace{1em} \frac{p}{q}
\le \frac{a}{b} \hspace{1em} \text{imply} \hspace{1em}
\frac{a}{b} = \frac{p}{q}\text{?}
\]

Yes. $a/b$ and $p/q$ are then (possibly two different ways of writing) the same
rational number.

\begin{enumerate}[a)]
\item
If S is not the null set, then $\le$ is well-defined because all of the elements
of
the equivalence classes are the same rational number. Iff a rational number
(equivalence class in our construction) is less than or equal to a second
rational, then the corresponding ordered pair representations of those rationals
have that same relationship. 

If S is the null set, then the claim is true because $x \le y$ and $[x] \le
[y]$ are both always false for any $x$ and $y$ imaginable. False is always
the boolean equivalent of false, and iff is the same as double implication
or boolean equivalence.

\item
Here an equivalence class is simply the set of ordered pairs of naturals that
can be used to
write some particular positive rational. So $\le$ must be reflexive and
transitive
on our equivalence classes because, from grade school math, it is reflexive and
transitive on rational numbers.

\item
By definition of terms, a number cannot be both greater than and less than some
other number. It also cannot be both greater than and equal, or both less than
and equal, some other number. So if $A$ is greater than or equal $B$, and $A$
is less than or equal $B$, $A = B$.

\item
No, because in this construction a positive rational represents an equivalence
class in the set of ordered pairs, not an element. A positive rational has an
infinite number of ordered pair representations, not any particular one.

\item
Since we are defining anti-symmetry with equivalence classes, the definition we
want here is:

\[
x \le y \hspace{6pt} \text{or} \hspace{6pt} y \le x \hspace{6pt} \text{or}
\hspace{6pt} x \equiv y 
\]

\item
Generally, we know that given two rationals $x$ and $y$, either the first is
less than the second, the second is less than the first, or they are equal.
If two ordered pairs from our construction are not in the same equivalence
class, then they are unequal and one is less than the other. If two ordered
pairs from our construction represent the same rational number,
then they are in the same equivalence class. So we have that two ordered pairs,
by the usual ordering of the rationals given in the problem statement, will
either be $<$, $>$, or $\equiv$ to one another.

\item

\end{enumerate}
\section{}
Dividing a list into two lists about a pivot is a subproblem of quicksort.

The set of primes is a \emph{subset} of the set of naturals.

2-SAT is no harder than 3-SAT.

If $A$ is a subproblem of $B$, then $A$ can be no harder than $B$, because $A$
must be solved in order to solve $B$. None of the other 5 possible implications
hold. 

\section{}
The first relationship holds because we can recognize the complement of a set
by simply flipping the output of a recognizer for the set from $YES$ to $NO$
and vice versa. This means
that recognizing a set has the same degree as recognizing the complement of a
set. By similar reasoning, the second relationship holds as well.

\section{}
\begin{enumerate}[a)]
\item
Want to show that

\[
(A \le_{log} B \hspace{1em} \text{and} \hspace{1em} B \le_{log} C)
\hspace{1em} \rightarrow \hspace{1em} A \le_{log} C
\]

Since we can bound time as exponential in the amount of space required, we have
that log-space reductions are also polynomial time reductions, because a
logarithm exponentiated is a polynomial. We already know that polynomial time
reductions used in NP, for example, are transitive.

\item


\end{enumerate}
\section{}

\section{}

\section{}
\begin{enumerate}[a)]
\item
Substitute the given solution into the difference equation to obtain:
\[
\left(\frac{1-\beta}{\beta}\right)^n \hspace{6pt} \stackrel{?}{=} \hspace{6pt}
\beta \left(\frac{1-\beta}{\beta}\right)^{n+1} + (1-\beta)
\left(\frac{1-\beta}{\beta}\right)^{n-1} 
\]

\[
\left(\frac{1-\beta}{\beta}\right)^n \hspace{6pt} \stackrel{?}{=} \hspace{6pt}
(1-\beta)\left(\frac{1-\beta}{\beta}\right)^{n} + 
\beta \left(\frac{1-\beta}{\beta}\right)^{n}
\]

\[
\left(\frac{1-\beta}{\beta}\right)^n \hspace{6pt} \stackrel{?}{=} \hspace{6pt}
\left(\frac{1-\beta}{\beta}\right)^{n} -
\beta \left(\frac{1-\beta}{\beta}\right)^{n} +
\beta \left(\frac{1-\beta}{\beta}\right)^{n}
\]

\[
\left(\frac{1-\beta}{\beta}\right)^n \hspace{6pt} = \hspace{6pt}
\left(\frac{1-\beta}{\beta}\right)^n
\]

\item
Substitution into the solution gives that $p_0 = 1$, and that it is certain $n$
must equal 0. So the algorithm must terminate.

\item
The substition in a) holds for any $\beta \ne 0$, which shows that the
same solution is still valid here, and that the algorithm must terminate
eventually.

\end{enumerate}

\end{document}
