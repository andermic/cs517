\documentclass{article}
\usepackage[pdftex]{graphicx}
\usepackage{amsmath}
\usepackage{verbatim}
\usepackage{enumerate}
\usepackage{longdiv}
\usepackage{parskip}
\author{Michael Anderson}
\title{Homework 7}
\begin{document}
\maketitle
\center{CS517}
\center{Prof. Cull}\\
\flushleft
\newpage

\section{}
By Savitch's Theorem, if it takes $O(c^n)$ space to recognize a set for
some constant positive value of $c$, Then it
takes $O((c^n)^2)$ space to simulate non-determinism for that set. Let
$\hat c = c^2$:

\[
O((c^n)^2) = O((c^2)^n) = O(\hat c^n)
\]

Since $\hat c$ is also a constant positive value, we are still in ESPACE, and
$N-ESPACE = ESPACE$.

\section{}
The possible configurations of some Turing Machine are no more than the
Cartesian Product its state space, its possible tape contents, and the position
of the read/write head. If a Turing
Machine enters the same configuration twice while executing, then it is stuck
in a dead loop
since it must continue entering that configuration again and again. Therefore,
the runtime is either infinite or bounded by $|Q| \cdot |\Gamma|^{S(n)} \cdot
S(n)$. Since
$Q$ and $\Gamma$ are both constants for any particular TM, we have that the
runtime of any nondeterministic program is either infinite or exponential in the
amount of space it uses.

\section{}
By Problem 2 and Savitch's Theorem, we have the runtime as either infinite or
$O(c^{S(n)^2})$ for some constant $c$.

\section{}
\begin{enumerate}[a)]
\item
Long division:\\ \vspace{6pt} \hspace{1em}
\longdiv{36}{12}
\item
$36/12 = (3 \cdot 3 \cdot 2 \cdot 2)/(3 \cdot 2 \cdot 2) = 3$
\item
Suppose we code algorithms for the above methods, and give each method its own
subroutine, and name them $DIV1$ and $DIV2$. Then given some function that
requires 
division among natural numbers we can code the division as a nondeterministic
instruction that either calls $DIV1$ or $DIV2$.

\end{enumerate}

\section{}
The arguments for all three parts (a), (b), and (c) are very similar. We have
trivially that $N-RE \supseteq RE $, $N-coRE \supseteq coRE$, and $N-Recursive
\supseteq Recursive$, because a deterministic acceptor, rejector, or
recognizer is also a nondeterministic acceptor, rejector, or recognizer
respectively.

Any nondeterministic program or part of a program that finishes in a finite
amount of time can be simulated in a finite amount of time by a deterministic
program. To make the conversion to a deterministic program, at each
nondeterministic instruction simply use fork to create a thread for each
possible path that the program could take, and execute each thread, thereby
taking all of the possible paths.

Deterministic versions of nondeterministic programs are likely to have more, but
still finite, runtimes. Since no particular runtime (except for finite when
applicable) is specified for the classes RE, coRE, or Recursive, we then have
that $N-RE \subseteq RE $, $N-coRE \subseteq coRE$, and $N-Recursive \subseteq
Recursive$; and that $N-RE = RE $, $N-coRE = coRE$, and
$N-Recursive = Recursive$.

\section{}
\begin{enumerate}[(a)]
\item
Each transition in a DM represents all and only all of the possible transitions
that could
have been taken non-deterministically in M. So if the DM ends up in an accepting
state after reading a string, M could also end up in an accepting state after
reading the same string, and will with magic guessing.

\item
Each possible state transition in M is encoded into DM by definition, and any
set of M states that contain an accepting M state are consider an accepting
state in the DM, so any path taken through M that leads to an accepting state
must correspond to a path through DM that leads to an accepting state.

\item
Any time DM rejects a string, M must also reject it because any possible path to
an accepting state in M would be represented within DM. Any time M rejects a
string, DM must reject it also, because DM cannot take paths to accepting states
that do not exist in M. Since M and DM must both accept and reject the same
sets, they recognize the same set.
\end{enumerate}

\section{}

\begin{tabular}{l | l l}
DM & 0 & 1 \\
\hline
$q_0$ & $q_{1,2}$ & $q_1$ \\
$q_1$ & $q_1$ & $q_3$ \\
$q_{1,2}$ & $q_{1,2}$ & $q_3$ \\ 
$q_3$ & $q_3$ & $q_3$ \\
\end{tabular}

This DM has only four states and it accepts strings of length $\ge$ 2, if a '1'
symbol appears somewhere after the first digit of the string.

\section{}
Such a nondeterministic machine would simply stay in the start state until
the kth to the end position is reached, and then if a '1' symbol was found in
that position transition into the first state of a chain of k states, the last
of which is the only accepting state. After reaching the end of this chain, any
additional symbols read in would lead to a trap state, so that the accepting
state would only be reached if exactly k-1 symbols were read in after the '1'
symbol in the kth to the last position. All of this requires $k+2 = O(k)$
states, an initial state, the ``chain" of k states, and the trap state.

\begin{enumerate}[(a)]
\vspace{1em}
\item
Let $q_a$ be the only accepting state in M, and $q_t$ be the trap state:

\vspace{6pt}

\begin{tabular}{l | l l}
M & Not 1 & 1 \\
\hline
$q_0$ & $q_0$ & $q_0,q_1$ \\
$q_1$ & $q_{a}$ & $q_{a}$\\
$q_{a}$ & $q_{t}$ & $q_{t}$ \\ 
$q_{t}$ & $q_{t}$ & $q_{t}$ \\
\end{tabular}

\vspace{6pt}

\begin{tabular}{l | l l}
DM & Not 1 & 1 \\
\hline
$q_0$ & $q_0$ & $q_{0,1}$ \\
$q_{0,1}$ & $q_{0,a}$ & $q_{0,1,a}$ \\
$q_{0,a}$ & $q_{0,t}$ & $q_{0,1,t}$ \\
$q_{0,t}$ & $q_{0,t}$ & $q_{0,1,t}$ \\
$q_{0,1,a}$ & $q_{0,a,t}$ & $q_{0,1,a,t}$ \\
$q_{0,1,t}$ & $q_{0,a,t}$ & $q_{0,1,a,t}$ \\
$q_{0,a,t}$ & $q_{0,t}$ & $q_{0,1,t}$ \\
$q_{0,1,a,t}$ & $q_{0,a,t}$ & $q_{0,1,a,t}$ \\
\end{tabular}

\vspace{6pt}

Here there are 7 states in the DM, and only 4 states in M, not too significant
of a difference.

\vspace{1em}

\item
Use the same defs for $q_a$ and $q_t$.

\vspace{6pt}

\begin{tabular}{l | l l}
M & Not 1 & 1 \\
\hline
$q_0$ & $q_0$ & $q_0,q_1$ \\
$q_1$ & $q_{2}$ & $q_{2}$\\
$q_2$ & $q_{a}$ & $q_{a}$\\
$q_{a}$ & $q_{t}$ & $q_{t}$ \\ 
$q_{t}$ & $q_{t}$ & $q_{t}$ \\
\end{tabular}

\begin{tabular}{l | l l}
M & Not 1 & 1 \\
\hline
$q_0$ & $q_0$ & $q_{0,1}$ \\
$q_{0,1}$ & $q_{0,2}$ & $q_{0,1,2}$ \\
$q_{0,2}$ & $q_{0,a}$ & $q_{0,1,a}$ \\
$q_{0,a}$ & $q_{0,t}$ & $q_{0,1,t}$ \\
$q_{0,t}$ & $q_{0,t}$ & $q_{0,1,t}$ \\
$q_{0,1,2}$ & $q_{0,2,a}$ & $q_{0,1,2,a}$ \\
$q_{0,1,a}$ & $q_{0,2,t}$ & $q_{0,1,2,t}$ \\
$q_{0,1,t}$ & $q_{0,2,t}$ & $q_{0,1,2,t}$ \\
$q_{0,2,a}$ & $q_{0,a,t}$ & $q_{0,1,a,t}$ \\
$q_{0,2,t}$ & $q_{0,a,t}$ & $q_{0,1,a,t}$ \\
$q_{0,a,t}$ & $q_{0,t}$ & $q_{0,1,t}$ \\
$q_{0,1,2,a}$ & $q_{0,2,a,t}$ & $q_{0,1,2,a,t}$ \\
$q_{0,1,2,t}$ & $q_{0,2,a,t}$ & $q_{0,1,2,a,t}$ \\
$q_{0,1,a,t}$ & $q_{0,2,t}$ & $q_{0,1,2,t}$ \\
$q_{0,2,a,t}$ & $q_{0,a,t}$ & $q_{0,1,a,t}$ \\
$q_{0,1,2,a,t}$ & $q_{0,2,a,t}$ & $q_{0,1,2,a,t}$ \\

\end{tabular}

\vspace{6em}

Here there are 5 states in M, and 16 states in DM. The number of states in M is
growing linearly, while the number of states in DM is growing exponentially.
This makes sense, because any DM for this type of problem must remember which of
the last k digits read were 1's, so that at any time when the string ends the
DM will know whether the kth to last digit was a 1. It takes $2^k$ many
states to represent all of those possible combinations.

\end{enumerate}
\end{document}
