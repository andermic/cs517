\documentclass{article}
\usepackage[pdftex]{graphicx}
\usepackage{amsmath}
\usepackage{verbatim}
\usepackage{enumerate}
\author{Michael Anderson}
\title{Homework 6}
\begin{document}
\maketitle
\center{CS517}
\center{Prof. Cull}\\
\flushleft
\newpage

\section{}
If $S$ has a primitive recursive acceptor, then it can be used to build a
primitive recursive recognizer. By definition of primitive recursive, we can
calculate the maximum number of steps that the acceptor could possibly run if it
will ever halt by
multiplying the maximum number of steps in its bounded loop(s) by the maximum
number of iterations of its bounded loops for some input. So run such an
acceptor for some input up to the calculated maximum number of steps, and if it 
halts at some point and returns YES than the input is in $S$. If the acceptor
halts and returns NO, or if it has not yet halted, than the input is in
$\bar S$. The recognizer runs in a boundable amount of time and is therefore in
PRIM.

\vspace{1em}

Very much like in the last paragraph, we can build an exponential time
recognizer from an exponential time acceptor. By definition, an exponential time
acceptor will halt and return YES on any input of size $n$ in no greater than $c\alpha^n$ time for some constants $c$ and $\alpha$. So a recognizer should run 
the acceptor on $n$ for $c\alpha^n$ steps and return YES if the acceptor
halts and returns
YES, or return NO if the acceptor halts and returns NO or if it has not yet
halted after $c\alpha^n$ steps. The recognizer runs in no more than exponential
time in the size of $n$.

\vspace{1em}

Finally, one example of a class of sets that have have acceptors and not
recognizers is the HALT set, or any other set in RE-COMPLETE. By definition of
RE these sets have acceptors, but by Turing's diagonalization argument they
cannot have recognizers.

\section{}
Suppose we find that a graph has a Hamiltonian circuit. Then it must also have a
Hamiltonian path, because we can remove the last edge in any Hamiltonian circuit
to get a Hamiltonian path.

\vspace{1em}

Suppose we find that a graph $G$ has no Hamiltonian circuit. Perform the
following algorithm to check for a Hamiltonian path in $G$:

\begin{verbatim}
FOR EACH vertex V1 in G:
    FOR EACH vertex V2 in G:
        if there is no edge in G between V1 and V2:
            make a graph G' with all of the vertexes and edges of G, but that
             also contains an edge between V1 and V2:
            IF there is a Hamiltonian circuit in G':
                RETURN YES
RETURN NO
\end{verbatim}

This algorithm works because if there is such a $G'$ with a Hamiltonian circuit
then the corresponding path in $G$ without the (V1,V2) edge is a Hamiltonian
path. This algorithm requires at most
$n^2$ (polynomial) calls to a Hamiltonian circuit finder, making Hamiltonian
path no harder than Hamiltonian circuit.

\section{}
Suppose we find that a graph has no Hamiltonian path. It cannot have a
Hamiltonian circuit,
since a Hamiltonian circuit is always a Hamiltonian path plus one additional
edge traversal.

\vspace{1em}

Suppose we find that a graph has a Hamiltonian path. Given some polynomial time
algorithm that finds all such paths, we could then check each one to see if
the start and end vertexes of such paths share an edge, and if any of them have
this property then add this extra edge to make a Hamiltonian circuit. If none
of them have the property then there is no Hamiltonian circuit. The
problem is that I cannot figure out 
how to find all of the paths in polynomial time, using just the
decision version of Hamiltonian path.

\section{}
Starting at $V_1$ and applying Prim's algorithm gives the edges of the
minimum spanning tree as: 

\[
(v_1,v_9),(v_1,v_3),(v_9,v_8),(v_3,v_2),(v_8,v_6),
(v_6,v_7),(v_7,v_5),(v_5,v_4)
\]

So for a path that the salesman might
take based on this tree we have the following edges along with their weights:

\[
(v_1,v_9,1),(v_9,v_8,3),(v_8,v_6,4),(v_6,v_7,3),(v_7,v_5,1),(v_5,v_4,4),
(v_4,v_3,6),(v_3,v_2,4)
\]

For a total cost of $1+3+4+3+1+4+6+4=26$. Upon inspection it looks as though in
this example the algorithm has actually produced the optimal path.

\section{}

\section{}
Use $n$ (polynomial) number of invocations of the oracle to guess the proper
relabeling of each vertex in $G_1$ to $G_2$, and each edge in $G_1$ to an edge
in $G_2$. This is a non-deterministic
polynomial time algorithm in the number of vertexes and edges that solves the
problem, therefore it is in NP.

\vspace{1em}

Also, let $G_1 = (V_1,E_1)$ and let $G_2 = (V_2,E_2)$. Let $SOLUTION_V(v)$ and
$SOLUTION_E(e)$ be the relabelings given by the oracle to $V_1$ and $E_1$
respectively. The solution is correct if the following algorithm returns YES,
and incorrect if it returns NO:

\begin{verbatim}
IF SOLUTIONV is not 1-1 or onto from V1 to V2, or SOLUTIONE is not 1-1 or onto
 from E1 to E2
    RETURN NO
IF there exists some edge (v,v') in E1 such that 
 SOLUTIONE((v,v')) != (SOLUTIONV(v), SOLUTIONV(v'))
    RETURN NO

RETURN YES
\end{verbatim}

Since the steps of this algorithm are polynomial in either the number of
the vertexes or the number of the edges of $G$, as they simply require iterating
through each of them and performing constant time operations in each iteration,
any solution is verifiable in polynomial time, the other formulation of NP.

\section{}
This problem does not seem to be any harder than graph isomorphism, which is
known to be in NP, but has not been proven to be NP-complete. Any
regular expression (star-free or not) can be represented easily and exactly
by a transition graph, where edges represent the reading of some string
contained within the expression, and $\wedge$'s and $\vee$'s are represented by 
arcs. If edges are then labeled by the strings that must be read in to enable
the transition, then simply compare the nodes, edges, and their labels of
one regular expression transition graph to another, and if all three are
equivalent then the regular expressions are equivalent, and vice versa for non-
equivalence.

\end{document}
