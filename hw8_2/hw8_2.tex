\documentclass{article}
\usepackage[pdftex]{graphicx}
\usepackage{amsmath}
\usepackage{verbatim}
\usepackage{enumerate}
\usepackage{parskip}
\author{Michael Anderson}
\title{Homework 8 - Problems 8-10}
\begin{document}
\maketitle
\center{CS517}
\center{Prof. Cull}\\
\flushleft
\newpage

\section{8}
\begin{enumerate}[a)]
\item
It is possible to construct a linear bounded automaton that accepts this
language, so it is definitely a CSL. This is because squaring a number can be
done in space linear in the length of the number.

\item
This language is not context-free (and therefore not regular also), by the
pumping lemma for context free languages. Whether the two pumped substrings are 
all
$a$'s, or they are all $b$'s, or one of them has some $a$'s and some $b$'s,
the pumped string cannot maintain the property that the number of
$b$'s is the square of the number of $a$'s. This is because the number of $b$'s 
required for a string to be in the language is not linear in the number of $a$'s
in such a string.

\end{enumerate}

\section{9}
\begin{enumerate}[a)]
\item
It is not known whether $NLIN-SPACE=LIN-SPACE$, but if it is not then
$NLIN-SPACE$ is in $PSPACE$, by Savitch's Theorem.

\item
Compilers must perform the task of deciding whether strings belong to a given
language defined by a given grammar. Since the worst problems in $LIN-SPACE$
are
not known to be solvable in polynomial time, defining languages as CSLs would
make compilers impractically slow.
\end{enumerate}

\section{10}
\begin{enumerate}[a)]
\item

\end{enumerate}
\end{document}
