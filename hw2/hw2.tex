\documentclass{article}
\usepackage[pdftex]{graphicx}
\usepackage{amsmath}
\usepackage{verbatim}
\usepackage{enumerate}
\author{Michael Anderson}
\title{Homework 2}
\begin{document}
\maketitle
\center{CS517}
\center{Prof. Cull}\\
\flushleft
\newpage

\section{}
\begin{tabular}{l l l l l l}
& 0 & 1 & 2 & 3 & \ldots \\
$f_0$ & 0 & 0 & 0 & 0 & \ldots \\
$f_1$ & 1 & 1 & 1 & 1 & \ldots \\
$f_2$ & 2 & 2 & 2 & 2 & \ldots \\
$f_2$ & 3 & 3 & 3 & 3 & \ldots \\
. & . & . & . & .\\
. & . & . & . & .\\
. & . & . & . & .\\
\end{tabular}

\vspace{1em}

The diagonal function $d(0) = 0$, $d(1) = 1$, $d(2) = 2$, \ldots is simply the
non-constant function $d(n) = n$.

\section{}
Given that the  $g(x)$'s are actually different constant functions, then $d(n)$
maps to different values on each of its input, which by definition makes it
non-constant.

\vspace{1em}

As for the BIG question, if the constant functions $g_0, g_1, g_2, \ldots$
are listed in no particular
order (I assume that this is what the question is asking, or else it is
trivial), then we cannot predict the growth rate of a diagonal function.
Because the constant functions could be ordered randomly, there is not 
necessarily any relationship between $n$ and $d(n)$, $d(n)$ could be increasing
over some intervals of $n$, and decreasing in others. Therefore there is no way
to say that when $n$ is greater than some constant, it is in some $\Theta$ or
$\Omega$ class.

\section{}

Let $F(i) = p_i(x)$ be the computable function that enumerates the polynomials.
Let a diagonal function $d$ be defined as $d(n) = Perturb(p_n(n))$, where
$Perturb(x)$ is some computable function such that 
$\forall x, \hspace{2pt} Perturb(x) \ne x$.

\vspace{1em}

$d$ is computable, because it is representable as the composition of two
computable functions:
$d(n) = Perturb(F(n)(n))$. $d$ is not a polynomial because all of the
polynomials
are enumerated in the table, and given some arbitrary
polynomial $p_n$, $d(n) \ne p_n(n)$ by the definition of $Perturb$.

\section{}
All of the argument in (3) generalizes to arbitrary classes of functions.

Again suppose there is a computable $F(i) = f_i$ that enumerates all of the
members of the class. Let $d(n) = Perturb(p_n(n))$ where $Perturb$ is a
function that does not map any of its possible inputs to themselves.

$d$ is computable by the computation $Perturb(F(n)(n))$, and $d$ does not
belong to the class enumerated by $F(n)$, because $d(n) \ne f_n(n)$ for any $n$.

\section{}
Can prove the claim by exhaustion.

\vspace{1em}

Since $x^4 > 6$ if $|x| > 1$, for integer valued $x$,
we need only consider $x \in \{-1,0,1\}$.

\vspace{1em}

Since $6 -0 = 6$ does not have an integer root, in other words there is no 
integer $y$ that satisfies $y^2 = 6$, $x \ne 0$. Similarly, Neither does
$6 - 1 = 5$, so
$x \ne 1$ and $x \ne -1$. Therefore the equation has no integer solutions.

\vspace{1em}

Since
the set of integers is a superset of the set of natural numbers, there are also
no natural number solutions.

\section{}
From the notes, $Recursive$ is the set of sets with recognizers,
$RE$ is the set of sets with acceptors, and $coRE$ is
the set of sets with rejectors.

\vspace{1em}

Assume that there is a set $S$ that has both an acceptor and a rejector. Then
imagine a program that runs the acceptor for $S$ and the rejector for $S$ in
two threads, and returns YES if the acceptor thread halts and returns YES, and
returns NO if the rejector thread halts and returns NO. Such a program would be 
a
recognizer for $S$. So if $S$ has an acceptor and a rejector, it also has a
recognizer, and therefore $RE \cap coRE \subseteq Recursive$.

\vspace{1em}

Assume that there is a set $S$ that has a recognizer. Since 
by definition a recognizer halts
and returns YES for inputs in $S$ it is an acceptor, and since it halts and
returns NO for inputs not in $S$ it is a rejector. So if $S$ has a recognizer,
it also has an acceptor and a rejector, and therefore
$Recursive \subseteq RE \cap coRE$.

\[
RE \hspace{1pt} \cap \hspace{1pt} coRE \subseteq Recursive \hspace{1em}
\text{ and } \hspace{1em}
Recursive \subseteq RE \cap coRE \hspace{1em} \Longrightarrow 
\]

\[
Recursive = RE \cap coRE
\]

\section{}

\section{}
\begin{enumerate}[(a)]
\item
If for some input $f_1,f_2$ to IN-EQ-PRIM the correct answer is YES, in other
words that
$\exists v, f_1(v) \ne f_2(v)$, then an acceptor can be built that will return
YES, and therefore IN-EQ-PRIM is in RE. 

\vspace{1em}

Such an acceptor could simply
iterate through all possible values of $v$, compute each $f_1(v)$ and $f_2(v)$,
and halt and return YES when a $v$ is found such that $f_1(v) \ne f_2(v)$. This
is doable in finite time because if the correct answer is YES then
such a $v$ actually exists and will eventually be found by the acceptor,
and because $f_1 \in PRIM$ and $f_2 \in PRIM$ so they have finite runtime for
all inputs.

\item
Let $R_H(s)$ be a hypothetical recognizer that returns YES if $s$ is in the
halting set, and NO if $s$ is not in the halting set. Let $R_I(f_1, f_2)$ be a
hypothetical recognizer for the IN-EQ-PRIM problem. Let $f_p(x)$ be a program
that runs the first $x$ steps or instructions of a program $p$. $f_p(x)$ should
return HALT if $p$ finished executing in $x$ steps or less, and it should
return DID NOT HALT if $p$ did not finish after $x$ steps. Let $SR(x)$ be a
program that simply returns DID NOT HALT regardless of the input.

\vspace{1em}

Now we can define a recognizer for the halt set as
$R_H(s) = R_I(f_s(x), SR(x))$.
Since $R_I$ is a recognizer for IN-EQ-PRIM it can see if
$f_s(x) = SR(x)$ for all values of $x$, in which case it correctly returns 
NO... $s$ will never halt regardless of how big $x$ gets. If $s$ halts at some
point, $f_s(x)$ returns HALT for some values of $x$, and $f_s(x) \ne SR(x)$,
and $R_I$ correctly returns YES.

\item
Because we have shown through a diagonal argument that no recognizer can exist
for the HALT set, and because by (b) if a recognizer exists for IN-EQ-PRIM
then it could be used
to build a recognizer for the halt set, there can be no recognizer for
IN-EQ-PRIM.
\end{enumerate}

\end{document}
